\documentclass[a4paper,fleqn,usenatbib]{mnras}
%=========================================================================
\usepackage{amsmath}
\usepackage{amssymb}
\usepackage{graphicx}
\usepackage{grffile}
\usepackage{float}
\usepackage[dvips]{epsfig}
\usepackage{epsfig}
\usepackage{dblfloatfix}
\usepackage{color}
\usepackage{caption}
\usepackage{hyperref}
\usepackage{bm}
\usepackage[british]{babel}
%Non reposionated tables
\newcommand{\HI}{{\text{H\MakeUppercase{\romannumeral 1}}} }
\newcommand{\lya}{\ifmmode{{\rm Ly}\alpha}\else Ly$\alpha$\ \fi}
\newcommand{\kms}{\ifmmode\mathrm{km\ s}^{-1}\else km s$^{-1}$\fi}
\newcommand{\vrot}{\ifmmode v_{\mathrm{rot}}\else $v_{\mathrm{rot}}$~\fi}
\newcommand{\vout}{\ifmmode v_{\mathrm{out}}\else $v_{\mathrm{out}}$~\fi}
\newcommand{\tauh}{\ifmmode \tau_{\mathrm{H}}\else $\tau_{\mathrm{H}}$~\fi}
\newcommand{\vth}{\ifmmode v_{\mathrm{th}}\else $v_{\mathrm{th}}$~\fi}
\newcommand{\hatk}{\ifmmode \hat{k}\else $\hat{k}$~\fi}
\newcommand{\STD}{\ifmmode \mathrm{STD}\else $\mathrm{STD}$~\fi}
\newcommand{\SKW}{\ifmmode \mathrm{SKW}\else $\mathrm{SKW}$~\fi}
\newcommand{\BI}{\ifmmode \mathrm{BI}\else $\mathrm{BI}$~\fi}

\begin{document}

%=========================================================================
%		FRONT MATTER
%=========================================================================
\title[Supercluster characterization in cosmological simulations]{Supercluster characterization in cosmological simulations}  
\author[D.L. Paipa-Leon \& J.E. Forero-Romero]{
  David Leonardo  Paipa-Le\'on$^{1}$
  \thanks{dl.paipa10@uniandes.edu.co} \&
  Jaime E. Forero-Romero $^{1}$
  \thanks{je.forero@uniandes.edu.co}\\
  %%
  $^{1}$ Departamento de F\'isica, Universidad de los Andes, Cra. 1
  No. 18A-10 Edificio Ip, CP 111711, Bogot\'a, Colombia \\
}
%%%%%====MYMARK========
\maketitle
\begin{abstract}
%Para referencia -------
when the velocities of a group of galaxies converge to a single structure, they are said to be part of the same supercluster. This structure has quantifiable properties such as mass, volume, density and inertia, among others.
 We compare cosmological simulations of N-bodies with the results obtained by Tully et al. (2014)  when determining these characteristics in Laniakea, our local supercluster.
 With the data of a simulation the space is discretized and the accretion for each voxel is quantified in order to determine the large-scale structure formation. We implement the Watershed algorithm in the grid in order to reconstruct and segregate the superclusters present in the simulation. We found  that Laniakea is an atypical event in terms of shape, volume and mass in the framework of cosmological simulations.
%------------------


\end{abstract}

\begin{keywords}
%%PREGUNTAR
galaxies:ISM --- line:profiles --- radiative transfer --- methods: numerical
\end{keywords}


%=========================================================================
%		PAPER CONTENT
%=========================================================================

%*************************************************************************

\section{Introduction}
\label{sec:intro}

Recent advances in instrumentation have revealed the presence of gas
rotation on vastly different physical scales.
\citep[e.g.][]{Herenz2016}.  




Throughout the paper we use a thermal velocity for a neutral Hydrogen
gas of \vth $= 12.86$ \kms, which corresponds to a temperature of
$T=10^4$ K. 

% \begin{figure*}
% \centering
%     \includegraphics[width=0.48\textwidth]{doppler_shift_logtau6_theta90}
%     \includegraphics[width=0.48\textwidth]{doppler_shift_logtau6_theta0}
%   \caption{\textbf{Qualitative trends of changing outflow and
%       rotational velocity viewed perpendicular/paralell to the
%       rotation axis}.  
%     Here we fix $\tauh=10^6$. 
%     The six panels on the left correspond to $\theta=90^\circ$ and the
%     panels on the right to $\theta=0^{\circ}$
%     We vary \vrot increasing from left to right and \vout increasing
%     from top to bottom. 
%     The thin black line corresponds to the \lya line obtained with
%     CLARA without any rotation and the indicated outflow velocity.
%     The thick black line corresponds to CLARA's results including both
%     outflows and rotation.
%     The thick gray line shows the results of modifying the pure outflow
%     solution (thin black line) by the Doppler shift presented in Equation \ref{eq:shift_x} using the respective \vrot. 
%     \label{fig:doppler_shift}}
% \end{figure*}

% \begin{figure*}
% \begin{center}
% \includegraphics[height=0.25\textwidth]{line_characterization_std_vout5_logtau5.pdf}
% \includegraphics[height=0.25\textwidth]{line_characterization_std_vout25_logtau5.pdf}
% \includegraphics[height=0.25\textwidth]{line_characterization_std_vout50_logtau5.pdf}\\
% \includegraphics[height=0.25\textwidth]{line_characterization_std_vout5_logtau6.pdf}
% \includegraphics[height=0.25\textwidth]{line_characterization_std_vout25_logtau6.pdf}
% \includegraphics[height=0.25\textwidth]{line_characterization_std_vout50_logtau6.pdf}\\
% \includegraphics[height=0.25\textwidth]{line_characterization_std_vout5_logtau7.pdf}
% \includegraphics[height=0.25\textwidth]{line_characterization_std_vout25_logtau7.pdf}
% \includegraphics[height=0.25\textwidth]{line_characterization_std_vout50_logtau7.pdf}
% \end{center}
% \caption{he
%   \label{fig:standard_deviation}}
% \end{figure*}

% \begin{figure*}
% \begin{center}
% \includegraphics[height=0.25\textwidth]{line_characterization_skw_vout5_logtau5}
% \includegraphics[height=0.25\textwidth]{line_characterization_skw_vout25_logtau5}
% \includegraphics[height=0.25\textwidth]{line_characterization_skw_vout50_logtau5}\\
% \includegraphics[height=0.25\textwidth]{line_characterization_skw_vout5_logtau6}
% \includegraphics[height=0.25\textwidth]{line_characterization_skw_vout25_logtau6}
% \includegraphics[height=0.25\textwidth]{line_characterization_skw_vout50_logtau6}\\
% \includegraphics[height=0.25\textwidth]{line_characterization_skw_vout5_logtau7}
% \includegraphics[height=0.25\textwidth]{line_characterization_skw_vout25_logtau7}
% \includegraphics[height=0.25\textwidth]{line_characterization_skw_vout50_logtau7}
% \end{center}
% \caption{\textbf{Skewness trends.} Results for all the
%   Radiative Transfer simulations (in triangles) compares against the
%   Doppler Shift model (lines).
%   Follows the same layout as Figure \ref{fig:standard_deviation}. 
%   \label{fig:skewness}}
% \end{figure*}

% \begin{figure*}
% \begin{center}
% \includegraphics[height=0.25\textwidth]{line_characterization_bi_vout5_logtau5}
% \includegraphics[height=0.25\textwidth]{line_characterization_bi_vout25_logtau5}
% \includegraphics[height=0.25\textwidth]{line_characterization_bi_vout50_logtau5}\\
% \includegraphics[height=0.25\textwidth]{line_characterization_bi_vout5_logtau6}
% \includegraphics[height=0.25\textwidth]{line_characterization_bi_vout25_logtau6}
% \includegraphics[height=0.25\textwidth]{line_characterization_bi_vout50_logtau6}\\
% \includegraphics[height=0.25\textwidth]{line_characterization_bi_vout5_logtau7}
% \includegraphics[height=0.25\textwidth]{line_characterization_bi_vout25_logtau7}
% \includegraphics[height=0.25\textwidth]{line_characterization_bi_vout50_logtau7}
% \end{center}
% \caption{\textbf{Bimodality trends.} Results for all the
%   Radiative Transfer simulations (in triangles) compares against the
%   Doppler Shift model (lines). 
%   Follows the same layout as Figure \ref{fig:standard_deviation}. 
%   \label{fig:bimodality}}
% \end{figure*}

% \begin{figure*}
% \begin{center}
% \includegraphics[height=0.25\textwidth]{line_characterization_vi_vout5_vrot100_logtau5}
% \includegraphics[height=0.25\textwidth]{line_characterization_vi_vout25_vrot100_logtau5}
% \includegraphics[height=0.25\textwidth]{line_characterization_vi_vout50_vrot100_logtau5}\\
% \includegraphics[height=0.25\textwidth]{line_characterization_vi_vout5_vrot100_logtau6}
% \includegraphics[height=0.25\textwidth]{line_characterization_vi_vout25_vrot100_logtau6}
% \includegraphics[height=0.25\textwidth]{line_characterization_vi_vout50_vrot100_logtau6}\\
% \includegraphics[height=0.25\textwidth]{line_characterization_vi_vout5_vrot100_logtau7}
% \includegraphics[height=0.25\textwidth]{line_characterization_vi_vout25_vrot100_logtau7}
% \includegraphics[height=0.25\textwidth]{line_characterization_vi_vout50_vrot100_logtau7}
% \end{center}
% \caption{\textbf{Valley Intensity. } We show for each \tauh the dependency that
%   the viewing angle $\theta$ has on the line's the
%   valley intensity. $\vrot=100$\kms is fixed for all panels.
% 		\label{fig:valley_intensity}}
% \end{figure*}

% \begin{figure}
% \centering
%     \includegraphics[width=0.48\textwidth]{doppler}
%   \caption{\textbf{Spectra from receding/approaching sides of a toy
%       model LAE.}  These results correspond to the RT simulation with
%     $\vout=25$\kms, $\vrot=50$\kms, $\tauh=10^5$. 
%     The spectra were computed for a viewing angle of
%     $\theta=90^{\circ}$. This toy model illustrates to what extent spectra from
%     opposite sides of a galaxy have an imprint of the rotational
%     kinematics. 
%     \label{fig:doppler}}
% \end{figure}


\section{Theoretical Models}
\label{sec:theory}



\begin{equation}
	v_{x}=\frac{x}{R}\vout - \frac{y}{R}\vrot ,
	\label{eq:vx}
\end{equation}

\begin{equation}
	v_{y}=\frac{y}{R}\vout + \frac{x}{R}\vrot ,
	\label{eq:vy}
\end{equation}

\begin{equation}
	v_{z}=\frac{z}{R}\vout,
	\label{eq:vz}
\end{equation}
%

\subsection{Quantitative Trends}
\label{sec:quantitative}


\begin{equation}
\label{eq:std}
\STD = \sqrt{m_2},
\end{equation}

\begin{equation}
\label{eq:skw}
\SKW = \frac{m_3}{m_2^{3/2}},
\end{equation}

\begin{equation}
\label{eq:bi}
\BI = \mathrm{KURTOSIS} - \SKW^2 = \frac{m_4}{m_2^{2}} - \frac{m_3^2}{m_2^{3}},
\end{equation}
%


In the range of parameter space explore, this difference has as an
upper bound of $\sim 7\%$, $3\%$ and $\sim 2\%$ for
 $\tauh=10^5$, $10^6$ and $10^7$, respectively. 



\section{Discussion}
\label{sec:discussion}
.

\subsection{\lya Kinematic Maps}
\label{subsec:kinematic}

($\approx 80$kpc) LAE.
In more recent work
receding/approaching spectra is close to $\sim 25\ \kms$, which is
a fourth of the naively expected value of $2\vrot=100\ \kms$,
due to the fact that only a small fraction of the photons are emitted
at the extreme of the galaxy having the maximum rotational velocity of
$50\ \kms$. 



\section{Conclusions}
\label{sec:conclusions}



\bibliographystyle{mnras}
\bibliography{references}



\appendix

\section{Additional figures}
\label{sec:appendix}

\begin{figure*}
  \begin{center}
    \includegraphics[width=0.49\textwidth]{doppler_shift_logtau5_theta90}
    \includegraphics[width=0.49\textwidth]{doppler_shift_logtau7_theta90}
  \end{center}
  \caption{\textbf{Qualitative trends of changing outflow and
      rotational velocity.}
    Same layout as Figure \ref{fig:doppler_shift}. On the left:
    $\tauh=10^5$ and $\theta=90^\circ$; on the right:     this time  $\tauh=10^7$ and $\theta=90^\circ$.}
\end{figure*}

\end{document}
